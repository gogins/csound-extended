\documentclass[12pt,letterpaper,onecolumn]{scrartcl}
\usepackage{tocloft}
\usepackage{stdclsdv}
\usepackage{comment}
\usepackage{vmargin}
\usepackage{t1enc}
\usepackage{fancyvrb}
\usepackage{url}
\usepackage{calc}
\usepackage{array}
\usepackage{scrpage2}
\usepackage{graphicx}
\usepackage{color}
\usepackage{listings}
\usepackage[latin1]{inputenc}
\usepackage[english]{babel}
\usepackage{supertabular}
\usepackage[pdftex,
            pagebackref=true,
            colorlinks=true,
            linkcolor=blue,
            pdfpagelabels,
            pdfstartpage=3
           ]{hyperref}
\pagestyle{scrheadings}
\makeindex
\definecolor{LstColor}{cmyk}{0.1,0.1,0,0.025} 
\setcounter{tocdepth}{2}
\begin{document}
\lstset{language=c++,basicstyle=\ttfamily\scriptsize,commentstyle=\ttfamily\small,tabsize=8,breaklines,backgroundcolor=\color{LstColor},fontadjust=true,keepspaces=false,fancyvrb=true,showstringspaces=false,moredelim=[is][\textbf]{\\emph\{}{\}}}.
\title{A Module System for Csound}
\author{Michael Gogins \\ \texttt{michael.gogins@gmail.com}}
\maketitle

\mainmatter
\section{Introduction}
\label{sec:Introduction}

This document presents a system for writing Csound orchestras that I have developed over a number of years to support my work as a composer. This document comes with a complete Csound piece that illustrates all features of the system, \texttt{module\_system\_example.csd}. The main purposes of the system are:

\begin{enumerate}
	\item To enable the maximum possible reuse of Csound instrument and user-defined opcode definitions. Any instrument or opcode created in one piece can immediately be used without modification in any other piece whose score uses the same pfield conventions.
	\item To enable moving modules between various user interface and external control systems, such as CsoundQt or the Csound6 Android app. 
	\item To implement a comprehensive, flexible, and high-quality system of spatialization, including Ambisonic periphony, arbitrary speaker arrangements, spatial reverberation including diffuse and specular early reflections from various surfaces, and distance cues including attenuation, filtering, Doppler effects, and head-related transfer functions. The spatialization system is adapted from Jan Jacob Hofmann's excellent work. 
	
\end{enumerate}

\noindent These goals are achieved by strictly following, insofar as the Csound orchestra language permits, that fundamental principle of software engineering known as \emph{encapsulation} or \emph{data hiding}. 

A Csound instrument definition or user-defined opcode definition is called a \emph{module}. These modules are ``black boxes'' that expose only the following standard interfaces for interactions with the rest of Csound. As a result, modules may be defined in include files (\textbf{.inc} files) and \texttt{\#include}d as required by the Csound orchestra.

\begin{enumerate}
	\item Standard pfields. Please note, in comparison with my past convention for pfields, I have changed slightly the meaning of the spatial pfields to more closely follow the conventions of Csound's spatialization opcodes. Also note that the exact same set of pfields and instrument definitions may be used either for score-driven performance, or for real-time, MIDI-driven performance, if the \texttt{-\--midi-key=4 -\--midi-velocity=5} command-line options are used. The first three pfields are mandatory except for \texttt{alwayson} modules, which do not require any pfields.
		\begin{enumerate}
			\item Instrument number or MIDI channel number (may be a fraction).
			\item Start time in beats (by default, a beat is 1 second).
			\item Duration in beats.
			\item Pitch as MIDI key number (may be a fraction).
			\item Loudness as MIDI velocity number (may be a fraction).
			\item Phase of the audio signal in radians (in case the instrument instance is, e.g., synthesizing a single grain of sound; this can be useful for phase-synchronous overlapped granular synthesis).
			\item Cartesian $x$ coordinate in meters, running from in front of the listener to behind the listener.
			\item Cartesian $y$ coordinate in meters, running from the left of the listener to the right of the listener.
			\item Cartesian $z$ coordinate in meters, running from below the listener to above the listener.
		\end{enumerate}
	\item Standard outlet and inlet ports. All modules must send or receive audio signals only via the signal flow graph opcodes. The following block of code can be used without modification at the end of almost any instrument definition:
	
\begin{lstlisting}
absignal[] init 16
absignal, aspatialreverbsend Spatialize asignal, kfronttoback, klefttoright, kbottomtotop
outletv "outbformat", absignal
outleta "out", aspatialreverbsend
\end{lstlisting}

That is what enables the same modules to be used in any sort of Csound orchestra and for any audio output file format or speaker rig.The B-format encoded audio signal can be decoded to mono, stereo, 2-dimensional panning, or full 3-dimensional panning using first, second, or third order decoding. 

		\item Standard control variables. The modules themselves do not define or directly use input or output channels or zak variables. They use k-rate global control variables with the following naming convention: \texttt{gkInstrumentName\_variableName}. These variables must be declared just above the instr or opcode definition and initialized there with default values. Their value must be normalized to range between 0 and 1, as if they were VST parameters.
		
In addition to only using these standard interfaces, all modules that use function tables must define their own tables within themselves using the \texttt{ftgenonce} opcode. Then there is no dependence of the module upon the external score or orchestra header.
		
Usually, a modular Csound piece will generate a score in the orchestra header, \texttt{\#include} a number of instrument and effects processing patches, and connect their outlets and inlets using the signal flow graph connect opcode. The signal flow graph will terminate in a module that will perform Ambisonic decoding to the specified speaker rig and/or output soundfile.

Finally, any external controls, such as CsoundQt widgets, OSC signals, MIDI controllers, or whatever, will be received an another custom input module and mapped to the relevant global control variables.
		
\end{enumerate}
\section{An Example}
\label{sec:AnExample}

\end{enumerate}

\bibliographystyle{unsrt}

\newpage
\bibliography{csound}

\end{document} 

